\chapter{Conclusioni}
\label{chap:conclusioni}
Attraverso il percorso di progettazione ed implementazione affrontato nel \autoref{chap:tre} e nel \autoref{chap:quattro}, si è giunti all'implementazione di un prototipo che mostri le principali componenti di una cloud application nel mondo dell'Internet of Things.\\
Il contributo che si è cercato di dare con questo lavoro di tesi è stato quello di mostare il processo logico e le motivazioni che abbiano portato ad alcune scelte progettuali che fossero poi riproducibili non solo in una particolare istanza, ma in tutte quelle applicazioni legate al mondo dell'IoT che condividano le stesse necessità e che, a prescindere dal loro scopo, abbiano una struttura comune.\\
La struttura, appunto, che è stata progettata ed implementata è sostanzialmente riconducibile a tre componenti fondamentali:
\begin{itemize}
	\item \textbf{Client Application:} Ovvero la applicazione e l'interfaccia che raccoglie e mostra i dati. In questo lavoro di tesi, l'interfaccia è un applicativo per smartphone Android. Tuttavia, le operazioni effettuate da questo applicativo sono eseguibili anche da qualsiasi altro dispositivo connesso in rete. Infatti, il suo compito sarà quello di raccogliere dati, inviare e ricevere dati attraverso Internet ed infine mostrare i dati. Si può pensare ad infinite altre soluzioni che siano in grado di riprodurre questi compiti e che quindi possano sostituire la client application implementata in questo lavoro di tesi per addattarsi a nuove necessità.
	
	\item \textbf{Server Application:} Ovvero la architettura del Server che è in ascolto e riceve i dati inviati dai client. In questo lavoro di tesi, per l'implementazione della Server application ci si è affidati ad un servizio esterno fornito da Amazon Web Services chiamato IoT Core. Anche in questo caso, i compiti che la server application deve svolgere sono, nella maggior parte, riproducibili da qualsiasi altro servizio cloud offerto da altre compagnie e perfino da servizi cloud implementati privatamente. I compiti ai quali dovrà assolvere una server application saranno infatti: autenticazione della comunicazione, implementazione di uno standard di comunicazione comune con i client, implementazione di una comunicazione publisher/subscriber con il client ed infine automatizzazione del processo di salvataggio dei dati ricevuti all'interno di un database.
	
	\item \textbf{Database Architecture and Data Analysis:} Strettamente collegata alla server application vi è anche la architettura del database nella quale memorizzare i dati raccolti ed eventualmente già filtrati dalla server application. Anche in questo caso, per l'implementazione di questo servizio, ci si è affidati ad un servizio esterno fornito da Amazon Web Services chiamato DynamoDB. Ma ancora, i compiti assolti da questo servizio sono riproducibili da altri servizi offerti da altre compagnie e perfino da databases gestiti localmente e privatamente. Infatti, il Database dovrà : consentire l'inserimento di nuovi elementi, consentire l' interrogazione di elementi già presenti, fornire delle interfacce per l'analisi dei dati contenuti al suo interno. 
\end{itemize}

Proprio focalizzandosi su queste poche componenti dell'architettura di una cloud application, si è dato un esempio implementativo di ognuna per lo sviluppo del sistema di raccolta dati relativi ad eventi stradali. \\
Sebbene si renda necessaria una successiva fase di 'ingegnerizzazione' di questo prototipo, i suoi utilizzi possono avere un impatto molto forte sul settore dell'automotive, già soggetto ad una vera e propria rivoluzione. Con una applicazione simile a quella prototipata in questo lavoro di tesi, si potrebbe infatti avere una sorta di scatola nera in ogni veicolo che attraversa la rete stradale e che quindi raccoglie miliardi di dati in frazioni di secondo. Ulteriormente, l'unione di questi dati e la loro analisi attraverso algoritmi di Machine Learning, potrebbero suppoprtare lo sviluppo e l'efficacia delle auto a guida autonoma. Tutto questo, al costo di uno smartphone.\\

